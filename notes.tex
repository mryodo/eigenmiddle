% DOC xelatex

\documentclass{mynotes}


\title{Interpolation of EigenSpaces in Graph/SC-induced operators}

\author[1]{ Tony Savostianov }

\affil[1]{ RWTH Aachen   \\ email: \email{anton.savostianov@gssi.it} }

\abstract{There are the notes for the interpolation of the eigenspaces for graph-induced operators.}

\keywords{interpolation, graph Laplacian, Riemann geometry}

\input{shortcuts.tex}

\begin{document}



\maketitle


\section{Initial definitions}

\begin{definition}[Parametric family]
      Let \( p \in [0; 1] \) be a real parameter for a family of square matrices \( A(p) \in \ds R^{n \times n} \) (such that each element \( a_{ij} = a_{ij}(p) \)). In order to simplify the discussion, we assume that \( \forall p: A^\top (p) = A (p) \) and each \( A(p )\) has real spectrum.
\end{definition}

The problem we want to deal with is the question of eigenspace interpolation: indeed, let \( A(p) \) first have only \emph{simple} eigenvalues; then, we aim to find first \( k \) eigenvectors of \( A(p) \) for every \( p \) given several exact estimations at \( \{ p_1, \dots p_N \}\). We denote the matrix composed of \( k\) first unit eigenvalues by \( C(p) \in \ds R^{n \times k} \), so 
\begin{equation}
      \begin{aligned}
            C(p)^\top C(p) & = I_k \\
            A(p) C(p) & =  C(p) \Lambda(p)
      \end{aligned}
      \label{eq:eig}
\end{equation}
such that \( \Lambda(p) = \diag \{ \lambda_1, \dots \lambda_k \} \) with \( \lambda_1 > \lambda_2 > \dots > \lambda_k \).

We also adopt the convention \( X ( p_i ) = X_i \).



\section{Two point problem}

Let us assume we have only two point estimation \( ( A_0, C_0) \) and \( (A_1, C_1 ) \). How can we estimate \( C_{\alpha}\)?
\begin{itemize}[itemsep = -5pt]
      \item compute a polynomial estimation: \( C_\alpha = \alpha C_0 + (1 - \alpha ) C_1 \);
      \item one needs \( C_\alpha \) to uphold \Cref{eq:eig}.
\end{itemize}
What do we make of \Cref{eq:eig}?
\begin{equation}
      \begin{aligned}
            C_\alpha^\top C_\alpha & = \left( \alpha C_0 + (1 - \alpha ) C_1 \right)^\top \left( \alpha C_0 + (1 - \alpha ) C_1 \right) = \\
            & = \alpha^2 C_0^\top C_0 + 2 \alpha (1-\alpha) \Sym( C_0^\top C_1 ) + (1-\alpha)^2 C_1^\top C_1 = \\
            & = ( 1 - 2\alpha + 2 \alpha^2) I_k + 2 \alpha (1-\alpha) \Sym( C_0^\top C_1 )  = \\
            & = I_k + 2 \alpha (1-\alpha) \Sym( C_0^\top C_1 - I_k )
      \end{aligned}
\end{equation}
Additionally,
\begin{equation}
      \begin{aligned}
            A_\alpha C_\alpha & = A_\alpha  \left( \alpha C_0 + (1 - \alpha ) C_1 \right) =  \alpha A_\alpha C_0 + (1 - \alpha ) A_\alpha C_1
      \end{aligned}
\end{equation}
This is clearly not enough: we need \( A_\alpha C_\alpha = C_\alpha \Lambda_\alpha \) such that \( \Lambda_\alpha \) is diagonal. Instead, we can write \( C_\alpha^\top A_\alpha C_\alpha = \Lambda_\alpha \); so we would need \( C_\alpha^\top A_\alpha C_\alpha \) be as close to diagonal as possible:
\begin{equation}
      \begin{aligned}
            C_\alpha^\top A_\alpha C_\alpha & = \left( \alpha C_0 + (1 - \alpha ) C_1 \right)^\top A_\alpha \left( \alpha C_0 + (1 - \alpha ) C_1 \right) = \\
            & = \alpha^2 C_0^\top A_\alpha C_0 + (1-\alpha)^2 C_1^\top A_\alpha C_1 + \\
            & + 2 \alpha (1-\alpha) \Sym( C_0^\top A_\alpha C_1 )
      \end{aligned}
\end{equation}
Let \( \Delta A_0 = A_\alpha - A_0 \) and \( \Delta A_1 = A_\alpha - A_1 \), so:
\begin{equation}
      \begin{aligned}
            C_0^\top A_\alpha C_0 & = C_0^\top \Delta A_0 C_0 + \Lambda_0 \\
            C_1^\top A_\alpha C_1 & = C_1^\top \Delta A_0 C_1 + \Lambda_1 
      \end{aligned}
\end{equation}
Can we inject something into \( C_0^\top A_\alpha C_1 + C_1^\top A_\alpha C_0 \)?
\begin{equation}
      \begin{aligned}
            \Sym( C_0^\top A_\alpha C_1 ) & = \Sym( C_0^\top A_0 C_1 ) + \Sym( C_1^\top \Delta A_0 C_0 ) \\
            \Sym( C_0^\top A_\alpha C_1 ) & = \Sym( C_0^\top A_1 C_1 ) + \Sym( C_1^\top \Delta A_1 C_0 )
      \end{aligned}
\end{equation}




\newpage
%% BIBLIOGRAPHY %% 
\nocite{*}
\bibliographystyle{alpha}
\bibliography{notes}



\end{document}