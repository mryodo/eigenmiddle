% DOC xelatex

\documentclass{mynotes}


\title{Interpolation of EigenSpaces in Graph/SC-induced operators}

\author[1]{ Tony Savostianov }

\affil[1]{ RWTH Aachen   \\ email: \email{anton.savostianov@gssi.it} }

\abstract{There are the notes for the interpolation of the eigenspaces for graph-induced operators.}

\keywords{interpolation, graph Laplacian, Riemann geometry}

\usepackage{tikz-network}
\usepackage{faktor}
\usepackage{comment}

\usepackage{algorithm}% http://ctan.org/pkg/algorithms
\usepackage{algpseudocode}
\graphicspath{./figures/}

\usepackage{tikz}
\usepackage{pgfplots}

\usetikzlibrary{arrows.meta}
\usetikzlibrary{backgrounds}
\usepgfplotslibrary{patchplots}
\usepgfplotslibrary{fillbetween}
\pgfplotsset{%
     layers/standard/.define layer set={%
         background,axis background,axis grid,axis ticks,axis lines,axis tick labels,pre main,main,axis descriptions,axis foreground%
     }{
         grid style={/pgfplots/on layer=axis grid},%
         tick style={/pgfplots/on layer=axis ticks},%
         axis line style={/pgfplots/on layer=axis lines},%
         label style={/pgfplots/on layer=axis descriptions},%
         legend style={/pgfplots/on layer=axis descriptions},%
         title style={/pgfplots/on layer=axis descriptions},%
         colorbar style={/pgfplots/on layer=axis descriptions},%
         ticklabel style={/pgfplots/on layer=axis tick labels},%
         axis background@ style={/pgfplots/on layer=axis background},%
         3d box foreground style={/pgfplots/on layer=axis foreground},%
     },
 }

 \usetikzlibrary{decorations.pathreplacing,decorations.markings}
 \tikzset{
   % style to apply some styles to each segment of a path
   on each segment/.style={
     decorate,
     decoration={
       show path construction,
       moveto code={},
       lineto code={
         \path [#1]
         (\tikzinputsegmentfirst) -- (\tikzinputsegmentlast);
       },
       curveto code={
         \path [#1] (\tikzinputsegmentfirst)
         .. controls
         (\tikzinputsegmentsupporta) and (\tikzinputsegmentsupportb)
         ..
         (\tikzinputsegmentlast);
       },
       closepath code={
         \path [#1]
         (\tikzinputsegmentfirst) -- (\tikzinputsegmentlast);
       },
     },
   },
   % style to add an arrow in the middle of a path
   mid arrow/.style={postaction={decorate,decoration={
         markings,
         mark=at position .5 with {\arrow[#1]{stealth}}
       }}},
 }
 


%%%% COMMANDS %%%%%%%%%%%%%%%%%%%%%%%%

\newcommand*{\insidefigure}[3][0.5\columnwidth]{
      \begin{center}
            \begin{minipage}{#1}
                  \centering
                  #2
                  \captionof{figure}{#3}
            \end{minipage}
      \end{center}
}

\newcommand*{\mc}[1]{\mathcal{#1}}
\renewcommand*{\bar}[1]{\overline{#1}}
\renewcommand*{\b}[1]{\mathbf{#1}}
\newcommand*\eps{\varepsilon}
\newcommand*{\V}[1]{ \mc V_{#1}(\mc K)}
\newcommand*{\vn}{\varnothing}
\newcommand*{\ord}[1]{\mathrm{ord}\,(#1)}

\usepackage{dsfont}
\newcommand*{\ds}[1]{\mathds{#1}}

\newcommand*{\Lu}[1]{L_{#1}^{\uparrow}}
\newcommand*{\bLu}[1]{{\overline{L}^{\uparrow}}_{#1}}
\newcommand*{\Ld}[1]{L_{#1}^{\downarrow}}
\newcommand*{\bLd}[1]{{\overline{L}^{\downarrow}}_{#1}}

\newcommand*{\wh}[1]{\widehat{#1}}


\renewcommand*{\bar}[1]{ \overline{#1} }


\newcommand{\algname}{\texttt{HeCS}}


\DeclareMathOperator{\im}{im}
\let\span\relax
\DeclareMathOperator{\span}{span}
\DeclareMathOperator{\Sym}{Sym}
\DeclareMathOperator{\Tr}{Tr}








\usetikzlibrary{patterns}
\definecolor{bananamania}{rgb}{0.98, 0.91, 0.71}
\definecolor{lavender}{rgb}{0.4470588235294118, 0.5294117647058824, 0.992156862745098}
\definecolor{burntsienna}{rgb}{0.91, 0.45, 0.32}
\definecolor{airforceblue}{rgb}{0.36, 0.54, 0.66}
\definecolor{liberty}{HTML}{5158BB}
\definecolor{junglegreen}{rgb}{0.16, 0.67, 0.53}


\newcommand{\AxisRotator}[1][rotate=0]{%
    \tikz [x=0.15cm,y=0.15cm,line width=.2ex,-stealth,#1] \draw (0,0) arc (-150:150:1 and 1);%
}
\newcommand{\AxisRotatorMirror}[1][rotate=0]{%
    \tikz [x=0.15cm,y=0.15cm,line width=.2ex,-stealth,#1] \draw (0,0) arc (150:-150:1 and 1);%
}
\newcommand{\EVert}[2]{%
\left[\begin{smallmatrix} #1 \\ #2 \end{smallmatrix}\right]
}
\newcommand{\TVert}[3]{%
\left[\begin{smallmatrix} #1 \\ #2 \\ #3 \end{smallmatrix}\right]
}

\usepackage{amsthm}
\usepackage[capitalize, nameinlink]{cleveref}
\crefdefaultlabelformat{\color{liberty}#1#2#3}


\crefname{section}{section}{sections}
\crefname{subsection}{subsection}{subsections}
\Crefname{section}{Section}{Sections}
\Crefname{subsection}{Subsection}{Subsections}
\Crefname{figure}{Figure}{Figures}
\Crefname{definition}{Definition}{Definitions}
\Crefname{proposition}{Proposition}{Propositions}
\Crefname{lemma}{Lemma}{Lemmas}
\Crefname{remark}{Remark}{Remarks}
\Crefname{problem}{Problem}{Problems}
\crefformat{equation}{\textup{#2(#1)#3}}
\crefrangeformat{equation}{\textup{#3(#1)#4--#5(#2)#6}}
\crefmultiformat{equation}{\textup{#2(#1)#3}}{ and \textup{#2(#1)#3}}
{, \textup{#2(#1)#3}}{, and \textup{#2(#1)#3}}
\crefrangemultiformat{equation}{\textup{#3(#1)#4--#5(#2)#6}}%
{ and \textup{#3(#1)#4--#5(#2)#6}}{, \textup{#3(#1)#4--#5(#2)#6}}{, and \textup{#3(#1)#4--#5(#2)#6}}

\Crefformat{equation}{#2Equation~\textup{(#1)}#3}
\Crefrangeformat{equation}{Equations~\textup{#3(#1)#4--#5(#2)#6}}
\Crefmultiformat{equation}{Equations~\textup{#2(#1)#3}}{ and \textup{#2(#1)#3}}
{, \textup{#2(#1)#3}}{, and \textup{#2(#1)#3}}
\Crefrangemultiformat{equation}{Equations~\textup{#3(#1)#4--#5(#2)#6}}%
{ and \textup{#3(#1)#4--#5(#2)#6}}{, \textup{#3(#1)#4--#5(#2)#6}}{, and \textup{#3(#1)#4--#5(#2)#6}}

\crefdefaultlabelformat{#2\textup{#1}#3}




%\newcommand{\algname}{\texttt{HeCS}}


\newtheorem{problem}{Problem}

\usepackage{enumitem}

\DeclareMathOperator{\diag}{diag} 

\newenvironment{mt}{\begin{pmatrix}}{\end{pmatrix}}

\usepackage{multirow}
\usepackage{booktabs}

\usepackage{nicematrix}

\NiceMatrixOptions
  {
    custom-line = 
     {
       letter = : ,
       command = dashedline , 
       ccommand = cdashedline ,
       tikz = dotted
     }
  }

  \setcounter{MaxMatrixCols}{20}
  \usepackage{amsmath}
  \allowdisplaybreaks
  
\usepackage{wasysym}
\usepackage{stackengine}

\makeatletter
\newcommand*\bigcdot{\mathpalette\bigcdot@{.5}}
\newcommand*\bigcdot@[2]{\mathbin{\vcenter{\hbox{\scalebox{#2}{$\m@th#1\bullet$}}}}}
\makeatother

  \newcommand\obullet[1]{\ensurestackMath{\stackon[0pt]{#1}{\mkern1mu\bigcdot}}}
%\renewcommand{\dot}{\obullet}

\DeclareMathOperator*{\vect}{\mathrm{diagvec}}
\DeclareMathOperator*{\grad}{\mathrm{grad}}
\DeclareMathOperator*{\diver}{\mathrm{div}}
\DeclareMathOperator*{\curl}{\mathrm{curl}}

\newcommand\mr{\min^+_{\mathrm{row}}}
\renewcommand*{\deg}{\mathrm{deg}}
\newcommand\scal[1]{\left\langle {#1} \right\rangle}

\usepackage[caption=false]{subfig}

\DeclareMathOperator{\argmin}{arg\,min}

%\usepackage{unicode-math}
%\usepackage{fontspec}

%\setmainfont{Zed Sans Light Extended}
%\setmainfont{Nunito Light}
%\setmainfont{Poppins Light}
%\setmathfont{Iwona Light}
%\setmathfont[Scale=1.5]{Nunito Light}%\setmathfont[Scale=1.2]

\begin{document}



\maketitle


\section{Initial definitions}

\begin{definition}[Parametric family]
      Let \( p \in [0; 1] \) be a real parameter for a family of square matrices \( A(p) \in \ds R^{n \times n} \) (such that each element \( a_{ij} = a_{ij}(p) \)). In order to simplify the discussion, we assume that \( \forall p: A^\top (p) = A (p) \) and each \( A(p )\) has real spectrum.
\end{definition}

The problem we want to deal with is the question of eigenspace interpolation: indeed, let \( A(p) \) first have only \emph{simple} eigenvalues; then, we aim to find first \( k \) eigenvectors of \( A(p) \) for every \( p \) given several exact estimations at \( \{ p_1, \dots p_N \}\). We denote the matrix composed of \( k\) first unit eigenvalues by \( C(p) \in \ds R^{n \times k} \), so 
\begin{equation}
      \begin{aligned}
            C(p)^\top C(p) & = I_k \\
            A(p) C(p) & =  C(p) \Lambda(p)
      \end{aligned}
      \label{eq:eig}
\end{equation}
such that \( \Lambda(p) = \diag \{ \lambda_1, \dots \lambda_k \} \) with \( \lambda_1 > \lambda_2 > \dots > \lambda_k \).

We also adopt the convention \( X ( p_i ) = X_i \).



\section{Two point problem}

Let us assume we have only two point estimation \( ( A_0, C_0) \) and \( (A_1, C_1 ) \). How can we estimate \( C_{\alpha}\)?
\begin{itemize}[itemsep = -5pt]
      \item compute a polynomial estimation: \( C_\alpha = \alpha C_0 + (1 - \alpha ) C_1 \);
      \item one needs \( C_\alpha \) to uphold \Cref{eq:eig}.
\end{itemize}
What do we make of \Cref{eq:eig}?
\begin{equation}
      \begin{aligned}
            C_\alpha^\top C_\alpha & = \left( \alpha C_0 + (1 - \alpha ) C_1 \right)^\top \left( \alpha C_0 + (1 - \alpha ) C_1 \right) = \\
            & = \alpha^2 C_0^\top C_0 + 2 \alpha (1-\alpha) \Sym( C_0^\top C_1 ) + (1-\alpha)^2 C_1^\top C_1 = \\
            & = ( 1 - 2\alpha + 2 \alpha^2) I_k + 2 \alpha (1-\alpha) \Sym( C_0^\top C_1 )  = \\
            & = I_k + 2 \alpha (1-\alpha) \Sym( C_0^\top C_1 - I_k )
      \end{aligned}
\end{equation}
Additionally,
\begin{equation}
      \begin{aligned}
            A_\alpha C_\alpha & = A_\alpha  \left( \alpha C_0 + (1 - \alpha ) C_1 \right) =  \alpha A_\alpha C_0 + (1 - \alpha ) A_\alpha C_1
      \end{aligned}
\end{equation}
This is clearly not enough: we need \( A_\alpha C_\alpha = C_\alpha \Lambda_\alpha \) such that \( \Lambda_\alpha \) is diagonal. Instead, we can write \( C_\alpha^\top A_\alpha C_\alpha = \Lambda_\alpha \); so we would need \( C_\alpha^\top A_\alpha C_\alpha \) be as close to diagonal as possible:
\begin{equation}
      \begin{aligned}
            C_\alpha^\top A_\alpha C_\alpha & = \left( \alpha C_0 + (1 - \alpha ) C_1 \right)^\top A_\alpha \left( \alpha C_0 + (1 - \alpha ) C_1 \right) = \\
            & = \alpha^2 C_0^\top A_\alpha C_0 + (1-\alpha)^2 C_1^\top A_\alpha C_1 + \\
            & + 2 \alpha (1-\alpha) \Sym( C_0^\top A_\alpha C_1 )
      \end{aligned}
\end{equation}
Let \( \Delta A_0 = A_\alpha - A_0 \) and \( \Delta A_1 = A_\alpha - A_1 \), so:
\begin{equation}
      \begin{aligned}
            C_0^\top A_\alpha C_0 & = C_0^\top \Delta A_0 C_0 + \Lambda_0 \\
            C_1^\top A_\alpha C_1 & = C_1^\top \Delta A_0 C_1 + \Lambda_1 
      \end{aligned}
\end{equation}
Can we inject something into \( C_0^\top A_\alpha C_1 + C_1^\top A_\alpha C_0 \)?
\begin{equation}
      \begin{aligned}
            \Sym( C_0^\top A_\alpha C_1 ) & = \Sym( C_0^\top A_0 C_1 ) + \Sym( C_1^\top \Delta A_0 C_0 ) \\
            \Sym( C_0^\top A_\alpha C_1 ) & = \Sym( C_0^\top A_1 C_1 ) + \Sym( C_1^\top \Delta A_1 C_0 )
      \end{aligned}
\end{equation}




\newpage
%% BIBLIOGRAPHY %% 
\nocite{*}
\bibliographystyle{alpha}
\bibliography{notes}



\end{document}